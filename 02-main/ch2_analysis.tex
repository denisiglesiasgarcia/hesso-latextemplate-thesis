\chapter{Analysis}
\label{chap:analysis}

\lipsum[1]

\minitoc

\newpage

% -----------------------------------------------------------------------------
This chapter shows example of picture and also serves to populate the different lists: list of figures, list of tables, bibliography, and glossary.

\section{Tables}

This section contains an examples of table: \autoref{tab:esempio}

\begin{table}[h]
	\centering
	\begin{tabular}{ccc}
		\toprule
		name & weight & food \\ 
		\midrule
		mouse	& 10 g	& cheese \\
		cat	& 1 kg	& mice \\
		dog	& 10 kg	& cats \\
		t-rex	& 10 Mg	& dogs \\
		\bottomrule 
	\end{tabular}
	\caption[A floating table]{A floating table.}
	\label{tab:esempio}
\end{table}

\section{Figures}

This section contains examples of figures: \autoref{fig:galleria}, \autoref{fig:lorem}, \autoref{fig:ipsum}, \autoref{fig:dolor}, \autoref{fig:sit}

\begin{figure}[h] 
	\centering 
	\includegraphics[width=0.5\columnwidth]{galleria_stampe} 
	\captionsource{A floating figure}{A floating figure: the lithograph \emph{Galleria di stampe}, of M.~Escher}{\url{http://www.mcescher.com/}}
	\label{fig:galleria} 
\end{figure}

\begin{figure}[h]
	\centering
	\begin{subfigure}[b]{0.45\textwidth}
		\includegraphics[width=\textwidth]{lorem}
		\caption{A gull}
		\label{fig:lorem}
	\end{subfigure}
	~ %add desired spacing between images, e. g. ~, \quad, \qquad, \hfill etc. 
	%(or a blank line to force the subfigure onto a new line)
	\begin{subfigure}[b]{0.45\textwidth}
		\includegraphics[width=\textwidth]{ipsum}
		\caption{A tiger}
		\label{fig:ipsum}
	\end{subfigure}
	~ %add desired spacing between images, e. g. ~, \quad, \qquad, \hfill etc. 
	%(or a blank line to force the subfigure onto a new line)
	\begin{subfigure}[b]{0.45\textwidth}
		\includegraphics[width=\textwidth]{dolor}
		\caption{A mouse}
		\label{fig:dolor}
	\end{subfigure}
	~ %add desired spacing between images, e. g. ~, \quad, \qquad, \hfill etc. 
	%(or a blank line to force the subfigure onto a new line)
	\begin{subfigure}[b]{0.45\textwidth}
		\includegraphics[width=\textwidth]{sit}
		\caption{A mouse}
		\label{fig:sit}
	\end{subfigure}
	\caption{Example subcaption}\label{fig:animals}
\end{figure}


% -----------------------------------------------------------------------------
\section{Code}

\autoref{lst:listing_example} shows an example of Java code rendered with minted.

\begin{listing}
	\javafile{02-main/listings/HelloWorld.java}
	\caption{Example of listing using the minted package}
	\label{lst:listing_example}
\end{listing}

% -----------------------------------------------------------------------------
\section{Other features}

Term (glossaries): \gls{nosql}

Acronym (glossaries): \gls{sql}

Citation (biblatex): \cite{paper_millwheel}

% -----------------------------------------------------------------------------
\section{Conclusion}

\blindtext
